\section{Anexos}

\begin{lstlisting} [language=Python,basicstyle=\tiny, label={code:generate_cscan},caption= {Codigo Python para generar trazas C }]
# Created By Luis Eduardo Quibano Alarcon

import os
import numpy as np
import random
import math
import sys
from user_libs.antennas.MALA import antenna_like_MALA_1200
from user_libs.antennas.GSSI import antenna_like_GSSI_1500
from gprMax.exceptions import CmdInputError
from gprMax.input_cmd_funcs import *
import json

userlibdir = os.path.dirname(os.path.abspath(__file__))

def cScanAntenna(upx, upy, upz, resolution, antenna_type, x_spacing,y_spacing,z_antenna,current_model_run,number_model_runs,inputfile, **kwargs):
    """

    Funcion utilizada para generar escenarios de trazas C dependiendo del tipo de antena a usar,
    este modulo funciona como extencion de gprMax para optimizar el proceso de simulacion

    Argumentos:
        upx, upy, upz :  valores en metros de las dimensiones x ,y,z del dominio
        resoution: resolucion en m del dominio (discretizacion) dependiendo de este valor se genera el PML automaticamente
        antenna_type:  tipo de antena , las opciones son: 'dipole' ,'MALA','GSSI'
        x_spacing, y_spacing: espacio que se movera la antena en x y y en cada iteración
        z_antenna: altura de la antena
        current_model_run: detecta el numero de la iteracion que esta corriendo actualmente
        number_model_runs: numero de iteraciones que va a tener el modelo
        inputfile:  ruta donde se esta ejecutando el modelo

        kwargs (dict): son valores opcionales, tales como la distancia entre atenas y polarizacion (para el caso de dipolo).
    """
    # upper right (x,y,z) coordinates
    #upx = 0.5
    #upy = 0.5
    #upz = 0.5

    # Resolution each coordinate
    dx = resolution
    dy = resolution
    dz = resolution

    # MINIMUM CELLS TO START
    #Minimo de celdas por encima de la fronteras  (Por defecto el PML es 10 celdas)
    startx = (dx * 14)
    starty = (dy * 14)
    startz = (dy * 14)

    #si la distancia entre antenas no se define, se utilizan 0.05m por defecto
    if "antennas_distance" in kwargs:
        antennas_distance = kwargs.get('antennas_distance', None)
    else:
        antennas_distance = 0.05

    last_xposition = round(upx - startx, 4)
    last_yposition = round(upy - starty, 4)
    last_zposition = round(upz - startz, 4)

    #x_spacing = 0.05  # STEP DISTANCE IN X
    #y_spacing = 0.05  # STEP DISTANCE IN Y
    #z_antenna = 0.4

    if (z_antenna > upz):
        print ("ALTURA ANTENA SUPERA EL DOMINIO")
        sys.exit()

    if (antenna_type == "dipole"):
        if "antenna_pol" in kwargs:
            antenna_pol = kwargs.get('antenna_pol', None)
        else:
            antenna_pol = "z"

        total_ascan = upx/x_spacing #Numero de ASCAN SIN PML SPACE

        # NUMERO TOTAL DE A SCANS TENIENDO EN CUENTA LOS PML
        ascans_per_bscan = math.floor( (last_xposition-(startx+antennas_distance)) / x_spacing)
        print("CADA " + str(ascans_per_bscan) + " A-SCANs ES UN B-SCAN")



        xoffset = dx  # OFFTSET IN X FOR CELLS SPACING
        yoffset = dy  # OFFTSET IN Y FOR CELLS SPACING

        bscan_num = np.floor((current_model_run - 1) / ascans_per_bscan)  # CurrentBSCAN NUMBER
        ascan_num = (current_model_run - 1) % ascans_per_bscan  # Current ASCAN NUMBER

        x = round(startx + (ascan_num * x_spacing), 4)  # CREATE X POSITION
        y = round(startx + (bscan_num * y_spacing), 4)  # CREATE Y POSITION

        bscan_num_final = np.floor((number_model_runs - 1) / ascans_per_bscan)  # CurrentBSCAN NUMBER
        ascan_num_final = np.ceil((number_model_runs - 1) % (ascans_per_bscan))  # Current ASCAN NUMBER

        final_x = round(startx + antennas_distance + ((ascan_num_final) * x_spacing), 4)  # FINAL POSITION IN X
        final_y = round(starty + (bscan_num_final * y_spacing), 4)  # FINAL POSITION IN Y
        print ("Ultima posicion de x: " + str(final_x))
        print ("Ultima posicion de y: " + str(final_y))

        data = {}
        data['antennaType'] = "Dipole"
        data['scansPerBScan'] = ascans_per_bscan
        data['finalAntennaPointX'] = final_x
        data['finalAntennaPointY'] = final_y
        data['number_model_runs'] = number_model_runs
        data['antenna_pol'] = antenna_pol
        data['inputfile'] = inputfile

        f = open(os.path.dirname(inputfile) + "/_Info_CSCAN.json", "w+")
        f.write(json.dumps(data))
        f.close()


        if ((final_y) > last_yposition):
            print("LA ANTENA SUPERA EL LIMITE DEL DOMINIO en Y que es " + str(last_yposition))
            print("Ultima posicion de x: " + str(final_x))
            print("Ultima posicion de y: " + str(final_y))
            print("Cambie el valor de n")
            sys.exit()
        else:
            print("hertzian_dipole: " + str(antenna_pol) + " " + str(x) + " " + str(y) + " " + str(z_antenna) + " my_ricker")
            print("rx: " + str( round(x + antennas_distance,4)) + " " + str(y) + " " + str(z_antenna) + " ")
            print("#hertzian_dipole: " + str(antenna_pol) + " " + str(x) + " " + str(y) + " " + str(z_antenna) + " my_ricker")
            print("#rx: " + str( round(x + antennas_distance,4)) + " " + str(y) + " " + str(z_antenna) + " ")

    if (antenna_type == "MALA"):

        # MALA DIMENSION
        # http://docs.gprmax.com/en/latest/user_libs_antennas.html?highlight=MALA
        xdimMALA = 0.184
        ydimMALA = 0.109
        zdimMALA = 0.046

        if (round(z_antenna + zdimMALA / 2, 4) > last_zposition):
            print("Posicion Z de la atena se sale del PLM")
            sys.exit()

        total_ascan = upx / x_spacing  # Numero de ASCAN SIN PML SPACE
        antnnapointX = round(startx + xdimMALA / 2, 4)  # OFFTSET IN X FOR CELLS SPACING
        antnnapointY = round(starty + ydimMALA / 2, 4)  # OFFTSET IN Y FOR CELLS SPACING

        xoffset = antnnapointX
        yoffset = antnnapointY

        # NUMERO TOTAL DE A SCANS TENIENDO EN CUENTA LOS PML
        ascans_per_bscan = math.floor((last_xposition - (xoffset + xdimMALA / 2)) / x_spacing)

        print("CADA " + str(ascans_per_bscan) + " A-SCANs ES UN B-SCAN")

        bscan_num = np.floor((current_model_run - 1) / ascans_per_bscan)  # CurrentBSCAN NUMBER
        ascan_num = current_model_run - (bscan_num * ascans_per_bscan) - 1  # Current ASCAN NUMBER
        print("current a scan: ", ascan_num, " bscan: ", bscan_num)

        xAntennaPoint = round(xoffset + (ascan_num * x_spacing), 4)  # CREATE X POSITION
        yAntennaPoint = round(yoffset + (bscan_num * y_spacing), 4)  # CREATE Y POSITION

        x = xAntennaPoint
        y = yAntennaPoint

        xAntennaEdgePoint = round(x + xdimMALA / 2, 4)
        yAntennaEdgePoint = round(y + ydimMALA / 2, 4)

        bscan_num_final = np.floor((number_model_runs - 1) / (ascans_per_bscan))  # CurrentBSCAN NUMBER
        ascan_num_final = number_model_runs - (bscan_num_final * (ascans_per_bscan)) - 1  # Current ASCAN NUMBER

        final_x = round(xoffset + (ascan_num_final * x_spacing), 4)  # FINAL POSITION IN X
        final_y = round(yoffset + (bscan_num_final * y_spacing), 4)  # FINAL POSITION IN Y
        final_xAntennaEdgePoint = round(final_x + xdimMALA / 2, 4)
        final_yAntennaEdgePoint = round(final_y + ydimMALA / 2, 4)

        data = {}
        data['antennaType'] = "MALA"
        data['scansPerBScan'] = ascans_per_bscan
        data['finalPositionXEdge'] = final_xAntennaEdgePoint
        data['finalPositionYEdge'] = final_yAntennaEdgePoint
        data['finalAntennaPointX'] = final_x
        data['finalAntennaPointY'] = final_y
        data['number_model_runs'] = number_model_runs
        data['inputfile'] = inputfile

        f = open(os.path.dirname(inputfile) + "/_Info_CSCAN.json", "w+")
        f.write(json.dumps(data))
        f.close()


        if ((round(final_yAntennaEdgePoint, 4)) > last_yposition):
            print("LA ANTENA SUPERA EL LIMITE DEL DOMINIO en Y que es " + str(last_yposition))
            print("Ultima posicion de x: " + str(final_x + xdimMALA / 2))
            print("Ultima posicion de y: " + str(final_y + starty))
            print("Cambie el valor de n")
            sys.exit()
        else:
            print("Ultima posicion de final_xAntennaEdgePoint: " + str(final_xAntennaEdgePoint))
            print("Ultima posicion de final_yAntennaEdgePoint: " + str(final_yAntennaEdgePoint))

        from user_libs.antennas.MALA import antenna_like_MALA_1200
        # print ("antenna_like_MALA_1200("+str(x)+", "+str(y)+", "+str(z_antenna)+", "+str(dx)+")")
        antenna_like_MALA_1200(x, y, z_antenna, dx)

    if (antenna_type == "GSSI"):

        # MALA DIMENSION
        # http://docs.gprmax.com/en/latest/user_libs_antennas.html?highlight=GSSI

        xdimGSSI = 0.170
        ydimGSSI = 0.108
        zdimGSSI = 0.045

        if (round(z_antenna+zdimGSSI / 2,4)>last_zposition):
            print ("Posicion Z de la Antena GSSI sobrepasa el PLM")
            sys.exit()


        total_ascan = upx / x_spacing  # Numero de ASCAN SIN PML SPACE
        antnnapointX = round(startx + xdimGSSI / 2, 4)  # OFFTSET IN X FOR CELLS SPACING
        antnnapointY = round(starty + ydimGSSI / 2, 4)  # OFFTSET IN Y FOR CELLS SPACING
        xoffset = antnnapointX
        yoffset = antnnapointY
        print(" antnnapointX", xoffset, " antnnapointY ", yoffset)
        # NUMERO TOTAL DE A SCANS TENIENDO EN CUENTA LOS PML
        ascans_per_bscan = math.floor((last_xposition - (xoffset + xdimGSSI / 2)) / x_spacing)

        print("CADA " + str(ascans_per_bscan) + " A-SCANs ES UN B-SCAN")

        bscan_num = np.floor((current_model_run - 1) / ascans_per_bscan)  # CurrentBSCAN NUMBER
        ascan_num = current_model_run - (bscan_num * ascans_per_bscan) - 1  # Current ASCAN NUMBER
        print("current a scan: ", ascan_num, " bscan: ", bscan_num)

        xAntennaPoint = round(xoffset + (ascan_num * x_spacing), 4)  # CREATE X POSITION
        yAntennaPoint = round(yoffset + (bscan_num * y_spacing), 4)  # CREATE Y POSITION

        x = xAntennaPoint
        y = yAntennaPoint

        xAntennaEdgePoint = round(x + xdimGSSI / 2, 4)
        yAntennaEdgePoint = round(y + ydimGSSI / 2, 4)

        bscan_num_final = np.floor((number_model_runs - 1) / (ascans_per_bscan))  # CurrentBSCAN NUMBER
        ascan_num_final = number_model_runs - (bscan_num_final * (ascans_per_bscan)) - 1  # Current ASCAN NUMBER

        final_x = round(xoffset + (ascan_num_final * x_spacing), 4)  # FINAL POSITION IN X
        final_y = round(yoffset + (bscan_num_final * y_spacing), 4)  # FINAL POSITION IN Y
        final_xAntennaEdgePoint = round(final_x + xdimGSSI / 2, 4)
        final_yAntennaEdgePoint = round(final_y + ydimGSSI / 2, 4)

        data = {}
        data['antennaType'] = "GSSI"
        data['scansPerBScan'] = ascans_per_bscan
        data['finalPositionXEdge'] = final_xAntennaEdgePoint
        data['finalPositionYEdge'] = final_yAntennaEdgePoint
        data['finalAntennaPointX'] = final_x
        data['finalAntennaPointY'] = final_y
        data['number_model_runs'] = number_model_runs
        data['inputfile'] = inputfile



        f = open(os.path.dirname(inputfile) + "/_Info_CSCAN.json", "w+")
        f.write(json.dumps(data))
        f.close()


        if ((round(final_yAntennaEdgePoint,4)) > last_yposition):
            print ("LA ANTENA SUPERA EL LIMITE DEL DOMINIO en Y que es " + str(last_yposition))
            print ("Ultima posicion de x: " + str(final_xAntennaEdgePoint))
            print ("Ultima posicion de y: " + str(final_yAntennaEdgePoint))
            print ("Cambie el valor de n")
            sys.exit()
        else:
            print ("Ultima posicion de final_xAntennaEdgePoint: " + str(final_xAntennaEdgePoint))
            print ("Ultima posicion de final_yAntennaEdgePoint: " + str(final_yAntennaEdgePoint))
            print("xAntennaEdgePoint", xAntennaEdgePoint, " yAntennaEdgePoint", yAntennaEdgePoint)
            print("xAntennaPoint ", x, "yAntennaPoint", y)

        # print ("BSCAN NUM:" + str(bscan_num))
        # print ("A SCAN NUM:" + str(ascan_num))
        # print (str(x+xdimGSSI/2) + "\n")
        # print (str(y+ydimGSSI/2) + "\n")
        from user_libs.antennas.GSSI import antenna_like_GSSI_1500
        # print("antenna_like_GSSI_1500(" + str(x) + ", " + str(y) + ", " + str(z_antenna) + ", resolution=" + str(dx) + ")")
        antenna_like_GSSI_1500(x, y, z_antenna, resolution=dx)
\end{lstlisting}
\newpage



\begin{lstlisting} [language=Python,basicstyle=\tiny, label={code:calc_iter},caption= {Codigo Python para calcular número de iteraciones }]
#!/usr/bin/env python
# -*- coding: utf-8 -*-
# Created By Luis Eduardo Quibano Alarcon
import os
import numpy as np
import random
import math
import sys
import ast
def calculate_dipole():
    stop=False
    number_model_runs = 1

    while (stop==False):
        total_ascan = upx / x_spacing  # Numero de ASCAN SIN PML SPACE
        # NUMERO TOTAL DE A SCANS TENIENDO EN CUENTA LOS PML
        # (10 CELDAS DE SEPARACION AL INICIO Y FINAL DEL DOMINIO)
        ascans_per_bscan = math.floor( (last_xposition-(startx+antennas_distance)) / x_spacing)

        xoffset = dx  # OFFTSET IN X FOR CELLS SPACING
        yoffset = dy  # OFFTSET IN Y FOR CELLS SPACING

        bscan_num = np.floor((number_model_runs - 1) / ascans_per_bscan)  # CurrentBSCAN NUMBER
        ascan_num = (number_model_runs - 1) % ascans_per_bscan  # Current ASCAN NUMBER

        final_x = round(startx + antennas_distance + ((ascan_num) * x_spacing), 4)  # FINAL POSITION IN X
        final_y = round(starty + (bscan_num * y_spacing), 4)  # FINAL POSITION IN Y
        if ((final_y) >= last_yposition):
            print ("    --------------------------------------------------")
            print ("    EL NUMERO n A ASIGNAR PARA EL DIPOLO ES "+str(number_model_runs-1))
            print ("    Agregar en el programa:")
            print ("#python:")
            print ("import os")
            print ("import numpy as np")
            print ("import random")
            print ("import math")
            print ("import sys")
            print ("from user_libs.antennas.CScan_Antennas import cScanAntenna")
            print ('cScanAntenna('+str(upx)+', '+str(upy)+', '+str(upz)+', '+str(resolution)+', "dipole", '+str(x_spacing)+', '+str(y_spacing)+', '+str(z_antenna)+', current_model_run, number_model_runs, inputfile, antennas_distance='+str(antennas_distance)+',antenna_pol='+str(antenna_pol)+')')
            print ("#end_python:")
            stop=True
        else:
            number_model_runs+=1


def calculate_MALA():
    stop=False
    number_model_runs = 1

    while (stop==False):
        # MALA DIMENSION
        # http://docs.gprmax.com/en/latest/user_libs_antennas.html?highlight=MALA
        xdimMALA = 0.184
        ydimMALA = 0.109
        zdimMALA = 0.046

        if (round(z_antenna+zdimMALA / 2,4)>last_zposition):
            print ("Posicion Z de la atena se sale del PLM")
            sys.exit()

        total_ascan = upx / x_spacing  # Numero de ASCAN SIN PML SPACE
        antnnapointX = round(startx + xdimMALA / 2, 4)  # OFFTSET IN X FOR CELLS SPACING
        antnnapointY = round(starty + ydimMALA / 2, 4)  # OFFTSET IN Y FOR CELLS SPACING

        xoffset = antnnapointX
        yoffset = antnnapointY

        # NUMERO TOTAL DE A SCANS TENIENDO EN CUENTA LOS PML
        # (10 CELDAS DE SEPARACION AL INICIO Y FINAL DEL DOMINIO)
        ascans_per_bscan = math.floor((last_xposition - (xoffset + xdimMALA / 2)) / x_spacing)

        bscan_num_final = np.floor((number_model_runs - 1) / (ascans_per_bscan))  # CurrentBSCAN NUMBER
        ascan_num_final = number_model_runs - (bscan_num_final * (ascans_per_bscan)) - 1  # Current ASCAN NUMBER

        final_x = round(xoffset + (ascan_num_final * x_spacing), 4)  # FINAL POSITION IN X
        final_y = round(yoffset + (bscan_num_final * y_spacing), 4)  # FINAL POSITION IN Y
        final_xAntennaEdgePoint = round(final_x + xdimMALA / 2, 4)
        final_yAntennaEdgePoint = round(final_y + ydimMALA / 2, 4)

        if ((final_yAntennaEdgePoint) > last_yposition):
            print ("    --------------------------------------------------")
            print ("    EL NUMERO n A ASIGNAR PARA ANTENA MALA ES " + str(number_model_runs - 1))
            print ("    Agregar en el programa:")
            print ("#python:")
            print ("import os")
            print ("import numpy as np")
            print ("import random")
            print ("import math")
            print ("import sys")
            print ("from user_libs.antennas.CScan_Antennas import cScanAntenna")
            print ('cScanAntenna(' + str(upx) + ', ' + str(upy) + ', ' + str(upz) + ', ' + str(
                resolution) + ', "MALA", ' + str(x_spacing) + ', ' + str(y_spacing) + ', ' + str(
                z_antenna) + ', current_model_run, number_model_runs, inputfile)')
            print ("#end_python:")
            stop = True
        else:
            number_model_runs += 1


def calculate_GSSI():
    stop = False
    number_model_runs = 1
    while (stop == False):
        # MALA DIMENSION
        # http://docs.gprmax.com/en/latest/user_libs_antennas.html?highlight=GSSI
        
        xdimGSSI = 0.170
        ydimGSSI = 0.108
        zdimGSSI = 0.045

        if (round(z_antenna+zdimGSSI / 2,4)>last_zposition):
            print ("Posicion Z de la atena se sale del PLM")
            sys.exit()

        total_ascan = upx / x_spacing  # Numero de ASCAN SIN PML SPACE
        antnnapointX = round(startx + xdimGSSI / 2, 4)  # OFFTSET IN X FOR CELLS SPACING
        antnnapointY = round(starty + ydimGSSI / 2, 4)  # OFFTSET IN Y FOR CELLS SPACING
        xoffset = antnnapointX
        yoffset = antnnapointY
        # NUMERO TOTAL DE A SCANS TENIENDO EN CUENTA LOS PML
        ascans_per_bscan = math.floor((last_xposition - (xoffset + xdimGSSI / 2)) / x_spacing)


        bscan_num = np.floor((number_model_runs - 1) / ascans_per_bscan)  # CurrentBSCAN NUMBER
        ascan_num = number_model_runs - (bscan_num * ascans_per_bscan) - 1  # Current ASCAN NUMBER

        xAntennaPoint = round(xoffset + (ascan_num * x_spacing), 4)  # CREATE X POSITION
        yAntennaPoint = round(yoffset + (bscan_num * y_spacing), 4)  # CREATE Y POSITION

        x = xAntennaPoint
        y = yAntennaPoint

        xAntennaEdgePoint = round(x + xdimGSSI / 2, 4)
        yAntennaEdgePoint = round(y + ydimGSSI / 2, 4)

        bscan_num_final = np.floor((number_model_runs - 1) / (ascans_per_bscan))  # CurrentBSCAN NUMBER
        ascan_num_final = number_model_runs - (bscan_num_final * (ascans_per_bscan)) - 1  # Current ASCAN NUMBER

        final_x = round(xoffset + (ascan_num_final * x_spacing), 4)  # FINAL POSITION IN X
        final_y = round(yoffset + (bscan_num_final * y_spacing), 4)  # FINAL POSITION IN Y
        final_xAntennaEdgePoint = round(final_x + xdimGSSI / 2, 4)
        final_yAntennaEdgePoint = round(final_y + ydimGSSI / 2, 4)

        if ((final_yAntennaEdgePoint) > last_yposition):
            print ("    --------------------------------------------------")
            print ("    EL NUMERO n A ASIGNAR PARA ANTENA GSSI ES " + str(number_model_runs - 1))
            print ("    Agregar en el programa:")
            print ("#python:")
            print ("import os")
            print ("import numpy as np")
            print ("import random")
            print ("import math")
            print ("import sys")
            print ("from user_libs.antennas.CScan_Antennas import cScanAntenna")
            print ('cScanAntenna(' + str(upx) + ', ' + str(upy) + ', ' + str(upz) + ', ' + str(
                resolution) + ', "GSSI", ' + str(x_spacing) + ', ' + str(y_spacing) + ', ' + str(
                z_antenna) + ', current_model_run, number_model_runs, inputfile)')
            print ("#end_python:")

            stop = True
        else:
            number_model_runs += 1

print ("##########################################################")
print ("CALCULAR NUMERO SIMULACIONES PARA TIPOS DE ANTENAS GPRMAX")
print ("1. Dipolo Herziano")
print ("2. Antena MALA")
print ("3. Antena GSSI")
case = ast.literal_eval(input("Ingrese Opción: "))
#case = 3
if (case==1):
    #antennas_distance = 0.05
    antennas_distance = ast.literal_eval(input("Distancia entre antenas: \n"))
    antenna_pol = ast.literal_eval(input("Direccion de polarizacion: \n"))

#dominio = [0.5, 0.5,0.5]
upx = ast.literal_eval(input("Ingrese dominio x: \n"))
upy = ast.literal_eval(input("Ingrese dominio y: \n"))
upz = ast.literal_eval(input("Ingrese dominio z: \n"))

#resolution=0.002
resolution = ast.literal_eval(input("Ingrese resolución : \n"))
dx=resolution
dy=resolution

#x_spacing=0.05
x_spacing= ast.literal_eval(input("Pasos en X : \n"))

#y_spacing=0.05
y_spacing=ast.literal_eval( input("Pasos en Y : \n"))

#z_antenna=0.4
z_antenna= ast.literal_eval(input("Altura Antena : \n"))

# MINIMUM CELLS TO START
plm= 14
startx = (dx * plm)
starty = (dy * plm)
startz = (dy * plm)
last_xposition=round(upx - startx,4)
last_yposition=round(upy - starty,4)
last_zposition = round(upz - startz, 4)

if (case==1):
    calculate_dipole()
elif(case==2):
    calculate_MALA()
elif (case == 3):
    calculate_GSSI()
else:
    sys.exit()
\end{lstlisting}

\newpage

\begin{lstlisting} [language=Python,basicstyle=\tiny, label={code:run_muliple},caption= {Codigo Python para ejecutar múltiples escenarios  en gprMax}]
import os
import numpy as np
import random
import math
import sys
import json
import glob
import subprocess
import argparse
#############################################################
# Created By Luis Eduardo Quibano Alarcon
# Los Andes University - Bogota, Colombia
# contact: le.quibano@uniandes.edu.co
#
# This script runs multiple models automaticaly by using a json configuration file
#
#############################################################
userlibdir = os.path.dirname(os.path.abspath('__file__'))

runGPU= False
gpuNumber=0

def str2bool(v):
  return v.lower() in ("yes", "true", "t", "1")

'''
Configuration file for running all models inside one path:
{
"multiFolder":false,
"directory": "/home/user/xxxxxxxx"
}
-------------------------------------
Configuration file for running specific models inside one path:
{
"multiFolder":true,
"directory": "/home/user/xxxxxxxx",
"models" : ["folder_model_1","folder_model_2","folder_model_3"]
}
 
 
'''

def runModels(runFile):


    if os.path.isfile(runFile):
        directoryToRun = open(runFile).read()
        toRun = json.loads(directoryToRun.replace('\\', '/'))
        if (str2bool(str(toRun["multiFolder"]))==False):
            if os.path.isdir(str(toRun["directory"]).rstrip()):
                ask =input("All models from  \n "+str(toRun["directory"]).rstrip()+ " will be simulated \n Are you sure? [y/n]: ")
                if (ask=="n" or ask=="N"):
                    sys.exit()
                elif (ask=="y" or ask=="Y"):
                    path = str(toRun["directory"]).rstrip()
                else:
                    sys.exit()

                askGPU= input("Use GPU? [y/n]: ")
                if (askGPU=="n" or askGPU=="N"):
                    runGPU= False
                elif (askGPU=="y" or askGPU=="Y"):
                    runGPU=True
                    gpuNumber = input("GPU Index: ")

                directories = os.listdir(path)

        elif (str2bool(str(toRun["multiFolder"])) == True):
            directories=[]
            askGPU = input("Use GPU? [y/n]: ")
            if (askGPU == "n" or askGPU == "N"):
                runGPU = False
            elif (askGPU == "y" or askGPU == "Y"):
                runGPU = True
                gpuNumber = input("GPU Index: ")
            print ("Multifolfer will be run")
            if os.path.isdir(str(toRun["directory"]).rstrip()):
                path = str(toRun["directory"]).rstrip()
                for folders in  toRun["models"]:
                    if os.path.isdir(str(toRun["directory"]).rstrip()+ "/"+folders):
                        directories.extend([folders])

        else:
            print("Configuration file does not have required parameters.")
            sys.exit()
    else:
        print("File do not exist: "+runFile)
        sys.exit()


    final_command=[]
    exist_in=False
    #RECORRO LAS CARPETAS DEL DIRECTORIO QUE ESTA CORRIENDO EL ARCHIVO
    for name in directories:
        if os.path.isdir(path+"/"+name):
            print("Checking Folder",name)
            #ABRO CADA CARPEYA
            file = glob.glob(path+"/"+name+'/*.in')
            #print(file)
            if (len(file)==1):
                #SI EN LA CARPETA EXISTE EL ARCHIVO .in CONTINUAR
                    if os.path.isfile(path+"/"+name+"/configure_model.json"):
                        json_data = open(path+"/"+name+"/configure_model.json").read()
                        data = json.loads(json_data)
                        #print(data)
                        if ( ('n' in data)  and ('geometry-only'in data)):
                            command=""
                            if (isinstance(data['n'],int) and isinstance(data['geometry-only'],bool) ):
                                command= "-n "+str(data['n'])
                                if (runGPU==True):
                                    command = command + " -gpu "+str(gpuNumber)
                                if (str2bool(str(data['geometry-only']))==True):
                                    command =command + " --geometry-only "
                                final_command.extend(["python -m gprMax "+str(file[0])+" "+command])
                                #print(final_command)
                            else:
                                print("parameters inside file are not correct")
                                sys.exit()

                        else:
                            print ("There is not parameters 'n' and 'geometry-only' inside "+path+"/"+name+"/configure_model.json")
                            sys.exit()
                    else:
                        print("Inside model folder  "+path+"/"+name+" do not exist configuration configure_model.json")
                        sys.exit()


            else:
                print("Inside model folder  " + path + "/" + name + " do not exist '.in' file or there are more than one '.in' files ")
                sys.exit()



    print ("----------------------")
    startRun=""
    i=1
    for command in final_command:
        if(i == len(final_command)):
            startRun= startRun + command+ " "
            #print (command )
        else:
            startRun = startRun + command  + " && "
            #print(command+ "&&")

        i=i+1

    start= input("Start Simulation? [y/n]: ")
    if (start=="Y" or start=="y"):
        print(startRun)
        subprocess.call(startRun, shell=True)
    elif (start=="N" or start=="n"):
        sys.exit()
    else:
        sys.exit()


if __name__ == "__main__":

    # Parse command line arguments
    parser = argparse.ArgumentParser(
        description='Run Multiple Simulations with Json file Congiguration.',
        usage='cd gprMax; python run_files.py  <Json PATH>')
    parser.add_argument('file', help='JSON file path')
    #parser.add_argument('--remove-files', action='store_true', default=False,
    #                   help='flag to remove individual output files after merge')

    args = parser.parse_args()

    #file = input("Configuration file path [Default=" + userlibdir + "/toRun.json /N]: ")


    if os.path.isfile(args.file.strip()):
        runFile = args.file.strip()
        runModels(runFile)
    else:
        print(file.strip() + " No se encontro")
        sys.exit()

\end{lstlisting}

\newpage


\begin{lstlisting} [language=Python,basicstyle=\tiny,label={code:merge_multipleAscan},caption= {Codigo Python para combinar trazas A}]
# Original Algorithm Created By:
# Copyright (C) 2015-2018: The University of Edinburgh
#                 Authors: Craig Warren and Antonis Giannopoulos
#
# Modified By : Luis Eduardo Quibano Alarcon
# Los Andes University
# Contact: le.quibano@uniandes.edu.co
# Description:  This File reads a folder passed by args and checks if _Info_Cscan.json is inside folder
#               After reads json file this script checks if number_model_runs from jsonfile is equal to modelrun's files
#               it creates N merged files according to  aScansPerBScan and  number_model_runs
import argparse
import glob
import os
import json
import sys
import h5py
import numpy as np
import math
import shutil
def merge_files(basefilename):
    """Merges traces (A-scans) from multiple output files into one new file,
        then optionally removes the series of output files.

    Args:
        basefilename (string): Base name of output file series including path.
        outputs (boolean): Flag to remove individual output files after merge.
    """
    if os.path.isdir((basefilename).rstrip()):
        nameInfo= ((basefilename+"/_Info_CSCAN.json").strip()).replace('\\', '/')
        if os.path.isfile(nameInfo):
            print ("Si Existe el _Info_CSCAN.json")
            json_data = open(nameInfo).read()
            data = json.loads(json_data)
            print (data)
            fileInput = data['inputfile'].split("/")
            fileNameInput= fileInput[-1]
            #print (fileInput[-1])
            #print(fileNameInput[0:-3])
            files = glob.glob(basefilename+"/"+fileNameInput[0:-3] + '*.out')
            outputfiles = [filename for filename in files if '_merged' not in filename]
            #print(outputfiles)
            modelruns = len(outputfiles)
            print("total archivos" ,modelruns)
            aScansPerBScan = data['scansPerBScan']
            print("aScansPerBScan",aScansPerBScan)
            if (data['number_model_runs']  != modelruns ):
                print("El numero de modelos no coincide con la informacion de archivo .json")
                sys.exit()

            totalBscans= math.ceil(modelruns / aScansPerBScan)
            print ("total Bscans",totalBscans)
            # Combined output file
            bscanResultFolder=basefilename+"/bscanResults"
            if os.path.isdir(bscanResultFolder):
                shutil.rmtree(bscanResultFolder, ignore_errors=True)
                os.makedirs(bscanResultFolder)
            else:
                os.makedirs(bscanResultFolder)

            for bscan in range(0,totalBscans):

                outputfile = bscanResultFolder+"/"+fileNameInput[0:-3]+"_Bscan"+str(bscan+1)+ '_merged.out'
                # Combined output file
                fout = h5py.File(outputfile, 'w')

                print("############")
                print("Creating ", outputfile)

                for model in range(0,aScansPerBScan):
                    #print(model)
                    modeltoopen =basefilename+"/"+fileNameInput[0:-3] + str(model + 1+aScansPerBScan*(bscan)) + '.out'
                    print("Adding ",modeltoopen)
                    fin = h5py.File(modeltoopen, 'r')
                    #print (fin)
                    nrx = fin.attrs['nrx']
                    #print ("(model+aScansPerBScan*(bscan)) ",(model+aScansPerBScan*(bscan)))
                    # Write properties for merged file on first iteration
                    if model == 0:
                        fout.attrs['Title'] = fin.attrs['Title']
                        fout.attrs['gprMax'] = "Modified By Luis"
                        fout.attrs['Iterations'] = fin.attrs['Iterations']
                        fout.attrs['dt'] = fin.attrs['dt']
                        fout.attrs['nrx'] = fin.attrs['nrx']
                        for rx in range(1, nrx + 1):
                            path = '/rxs/rx' + str(rx)
                            grp = fout.create_group(path)
                            availableoutputs = list(fin[path].keys())
                            for output in availableoutputs:
                                grp.create_dataset(output, (fout.attrs['Iterations'], aScansPerBScan),
                                                   dtype=fin[path + '/' + output].dtype)

                    # For all receivers
                    for rx in range(1, nrx + 1):
                        path = '/rxs/rx' + str(rx) + '/'
                        availableoutputs = list(fin[path].keys())
                        #print(availableoutputs)
                        # For all receiver outputs
                        for output in availableoutputs:
                            #print(fout[path + '/' + output][:, model])
                            fout[path + '/' + output][:, model] = fin[path + '/' + output][:]
                    fin.close()






                fout.close()
    else:
        print ("La ruta seleccionada no es un directorio")

if __name__ == "__main__":
    # Parse command line arguments
    parser = argparse.ArgumentParser(
        description='Merges traces (A-scans) from multiple output files into one new file, then optionally removes the series of output files.',
        usage='cd gprMax; python -m tools.outputfiles_merge basefilename')
    parser.add_argument('basefilename', help='base name of output file series including path')
    #parser.add_argument('--remove-files', action='store_true', default=False,
    #                   help='flag to remove individual output files after merge')
    args = parser.parse_args()

   # args.basefilename = os.getcwd()+"/merge_Cscan"
    #args.basefilename = "C:/Users/Luis/Desktop/pruebas/cscan_x_mala_2_barras/"
    print (args.basefilename)
    merge_files(args.basefilename)




\end{lstlisting}